%versi 3 (18-12-2016)
\chapter{File Docker Eksperimen}
\label{lamp:C}

%terdapat 2 cara untuk memasukkan kode program
% 1. menggunakan perintah \lstinputlisting (kode program ditempatkan di folder yang sama dengan file ini)
% 2. menggunakan environment lstlisting (kode program dituliskan di dalam file ini)
% Perhatikan contoh yang diberikan!!
%
% untuk keduanya, ada parameter yang harus diisi:
% - language: bahasa dari kode program (pilihan: Java, C, C++, PHP, Matlab, C#, HTML, R, Python, SQL, dll)
% - caption: nama file dari kode program yang akan ditampilkan di dokumen akhir
%
% Perhatian: Abaikan warning tentang textasteriskcentered!!
%


\begin{lstlisting}[language=docker-compose,caption=File \textit{docker-compose} yang digunakan untuk Experiment, label=kode:5:3:1:dockercompose]
version: "3"
services:
  codeigniter-3:
    build: .
    ports:
      - "81:80"
    volumes:
      - .:/var/www/html
    depends_on:
      - db

  db:
    image: mysql:5.7
    ports:
      - "3306:3306"
    environment:
      MYSQL_TABLE: judge
      MYSQL_USER: sharif
      MYSQL_PASSWORD: judge
      MYSQL_ROOT_PASSWORD: root
    command: --sql_mode=STRICT_TRANS_TABLES,NO_ZERO_IN_DATE,NO_ZERO_DATE,ERROR_FOR_DIVISION_BY_ZERO,NO_ENGINE_SUBSTITUTION
    volumes:
      - ./mysql:/var/lib/mysql

  phpmyadmin:
    image: phpmyadmin/phpmyadmin
    ports:
      - "8080:80"
    environment:
      PMA_HOST: db
      MYSQL_ROOT_PASSWORD: freehost
    depends_on:
      - db
\end{lstlisting}

\begin{lstlisting}[language=docker, caption=File \textit{Dockerfile} yang digunakan untuk Experiment, label=kode:5:3:1:dockerfile]
# Menggunakan image PHP 7.3 sebagai base image
FROM php:7.3-apache

# Install dependensi dan ekstensi PHP yang dibutuhkan untuk CodeIgniter
RUN apt-get update && apt-get install -y \
    libpng-dev \
    libjpeg-dev \
	libldap2-dev \
	libcurl4 \
    libcurl4-openssl-dev \
	libzip-dev \
    libfreetype6-dev \
    zip \
    unzip \
	default-jdk \
	g++ \
	python2 \
	python3

# Install ekstensi GD dan mysqli
RUN docker-php-ext-configure gd --with-freetype-dir=/usr/include/ --with-jpeg-dir=/usr/include/ \
    && docker-php-ext-install gd mysqli
	
RUN docker-php-ext-install curl

RUN  docker-php-ext-configure ldap --with-libdir=lib/x86_64-linux-gnu/ \
	&& docker-php-ext-install ldap

RUN docker-php-ext-install fileinfo
RUN docker-php-ext-install mbstring
RUN docker-php-ext-install zip

RUN cp /usr/local/etc/php/php.ini-production /usr/local/etc/php/php.ini && \
        sed -i -e "s/^ *memory_limit.*/memory_limit = 4G/g" /usr/local/etc/php/php.ini && \
        sed -i -e "s/^ *max_input_vars.*/max_input_vars = 3000000/g" /usr/local/etc/php/php.ini && \
        sed -i -e "s/^ *post_max_size.*/post_max_size = 50M/g" /usr/local/etc/php/php.ini && \
        sed -i -e "s/^ *upload_max_filesize.*/upload_max_filesize = 50M/g" /usr/local/etc/php/php.ini

# Aktifkan mod_rewrite untuk Apache
RUN a2enmod rewrite

# Copy kode CodeIgniter ke dalam container
COPY . /var/www/html/

# Set direktori kerja
WORKDIR /var/www/html/

# Make Folder tester writeable by PHP
RUN chmod 777 /var/www/html/restricted/tester
RUN chmod 777 /var/www/html/application/cache/Twig

# Expose port 80
EXPOSE 80

# Jalankan Apache server
CMD ["apache2-foreground"]
\end{lstlisting}

% \lstinputlisting[language=Java, caption=MyCode.java]{./Lampiran/MyCode.java} 

