\chapter{Implementasi dan Pengujian}
\label{chap:implementasidanpengujian}

Bab ini membahas mengenai implementasi dan pengujian sistem perekaman ulang dalam SharIF-Judge.

\section{Implementasi}
\label{sec:5:implementasi}

Bagian ini menjelaskan hasil implementasi sistem pemutaran ulang pada SharIF-Judge berdasarkan perancangan pada Bab \ref{chap:analisis}. Pada saat implementasi juga dilakukan penyesuaian pada peracangan yang sudah dibuat untuk mengatasi kendala yang dialami pada saat implementasi. 

\subsection{Merekam Peristiwa pada IDE}
\label{sub:5:2:merekam}

Fitur merekam peristiwa pada editor kode diimplementasikan untuk menangkap seluruh interaksi pengguna terhadap IDE dan SharIF-Judge pada saat pengguna menyelesaikan sebuah masalah dalam \textit{assignment}. Data yang direkam akan digunakan untuk memutar ulang penyelesaian yang dilakukan oleh pengguna. Fitur ini diimplementasikan dengan memanfaatkan \textit{Library Ace}, \textit{event hooks} pada javascript. Implementasi ini akan memerlukan penyesuian pada bagian \textit{javascript} yaitu file \verb|assets/js/shj_submit.js| yang dimuat pada halaman \textit{Submit}.

Berikut merupakan alur sistem perekaman peristiwa:
\begin{enumerate}
    \item Inisialisasi Perekam \\
    Pada alur ini, semua perekaman akan diinisialisasi dengan menjalankan sebuah fungsi dinamakan \verb|recordStart|. Fungsi ini akan memanggil seluruh \textit{event hooks} dan \textit{event listener} dalam \textit{Library ace} agar dinyalakan. 
    
    Dalam menjalankan sebuah fungsi \textit{event listener} dibutuhkannya dua argumen yaitu event yang akan di panggil dan sebuah \textit{callback function} yang akan dipanggil pada saat terjadinya sebuah event tersebut. 
    
    Dalam implementasi akan dibuat 2 buah \textit{object javascript} yang menjadi fungsi \textit{event listener} yaitu \textit{object} pemanggilan \textit{event listener} dan \textit{object} menyimpan \textit{callback function} yang dipakai oleh \textit{event listener} masing-masing. Kode \ref{kode:5:1:objcallback} merupakan \textit{object callback function} yang diimplementasikan.
    \begin{lstlisting}[caption={\textit{object callback function}}, label={kode:5:1:objcallback}]
const handlers = {
    editor_change: (e) =>
        recordEvent(e.action, {
            data: e.lines,
            start: e.start,
            end: e.end,
        }),
}
    \end{lstlisting}
    
    Kode \ref{kode:5:1:objeventlist} merupakan \textit{object} pemanggilan \textit{event listener} yang dipanggil pada saat inisialisasi dan mendapatkan fungsi \textit{callback function} dari \textit{object handlers} pada Kode \ref{kode:5:1:objcallback}.
    \begin{lstlisting}[caption={\textit{object event listener}}, label={kode:5:1:objeventlist}]
const addListener = {
    editor_change: () => editor.session.on("change", handlers.editor_change),
}
    \end{lstlisting}

    Setelah itu \textit{object addListener} akan dipanggil oleh fungsi \verb|recordStart|. Kode \ref{kode:5:1:eventlistener} menunjukan fungsi yang dipanggil saat inisialisasi. 
    
    \begin{lstlisting}[caption={Beberapa \textit{event listener} yang dipanggil}, label={kode:5:1:eventlistener}]
addListener.editor_change();
    \end{lstlisting}
    
    Fungsi \verb|recordStart| akan dilakukan pada saat pengguna mengubah \textit{problem} yang dipilih dalam halaman Submit.

    \item Penyimpan sebuah rekaman \\
    Untuk setiap perekaman yang dibutuhkan, dijalankan sebuah fungsi yang mendeteksi perubahan tersebut dan menjalankan sebuah \textit{callback function} saat terjadi perubahan tersebut. Fungsi ini dipnaggil pada saat inisialisasi. \textit{Callback function} tersebut akan mendapatkan argumen sesuai dengan perubahan yang dideteksi. Pada contohnya untuk perubahan teks pada editor kode yaitu fungsi \verb|onchange|, argumen yang diberikan merupakan teks yang dimasukan yaitu contohnya adalah `A', dan juga posisi dimana teks tersebut dimasukkan dalam editor kode. Kode \ref{kode:5:1:argonchange} merupakan contoh argumen yang diberikan.
    \begin{lstlisting}[caption={Contoh argumen yang diberikan oleh fungsi onchange}, label={kode:5:1:argonchange}]
{
    data: ["A"], 
    start: {row: 0, column: 1}, 
    end:{ row: 0, column: 2}
}
    \end{lstlisting}
    Setelah itu data akan disimpan dalam sebuah \textit{object javascript}
    seperti yang sudah dijelaskan pada Bab \ref{sec:4:2:storerekaman}. data argumen akan disimpan mengunakan key `args' atau `payload'.
    
    \item Penyimpanan data rekaman \\
    Selanjutnya sebuah \textit{event} atau rekaman yang sudah dicatat dan menjadi sebuah \textit{object javascript} bernama \verb|recording|. seluruh event rekaman akan simpan dalam sebuah \textit{array} dalam \verb|recording| dengan key `events' seperti yang sudah dijelaskan pada Bab \ref{sec:4:2:storerekaman}. \verb|recording| juga memiliki waktu dimulainya rekaman, isi awal editor kode, posisi awal cursor dalam editor kode. \verb|recording| juga memiliki fungsi \verb|init| untuk meninisialisasi seluruh \textit{object} \verb|recording|.

    \begin{lstlisting}[caption={Contoh argumen yang diberikan oleh fungsi onchange}, label={kode:5:1:recordingobj}]
const recording = {
    events: [],
    startTime: -1,
    startValue: "",
    startSelection: [],

    init: () => {
        recording.events = [];
        recording.startTime = Date.now();
        recording.startValue = editor.getValue();
        recording.startSelection = getSelection(editor);
    },
};
    \end{lstlisting}

    Kode \ref{kode:5:1:recordingobj} merupakan \verb|recording| pada saat keadaan kosong. Pada \textit{object} \verb|recording|, \verb|events| merupakan sebuah array dengan isi sebuah rekaman event, \verb|startTime| merupakan waktu awal rekaman dimulai, \verb|startValue| merupakan isi awal dalam editor kode, dan \verb|startSelection| merupakan posisi awal cursor dalam editor kode.
\end{enumerate}

Berikut merupakan peristiwa yang akan direkam oleh sistem perekaman:
\begin{itemize}
    \item \verb|editor.change|: Peristiwa ini akan menangkap perubahan isi teks dalam editor kode.
    \item \verb|editor.changeCursor|: Peristiwa ini akan menangkap perubahan cursor dalam editor kode. 
    \item \verb|editor.changeSelection|: Peristiwa ini akan menangkap perubahan selection cursor dalam editor kode.
    \item \verb|window.focus|: Peristiwa ini akan menangkap pengguna pada saat pengguna \textit{focus} pada SharIF Judge web page. 
    \item \verb|window.blur|: Peristiwa ini akan menangkap pengguna pada saat pengguna tidak \textit{focus} pada SharIF Judge web page atau \textit{focus} pada aplikasi lain atau web page lain.
    \item \verb|window.visibilitychange|: Peristiwa ini akan menangkap pengguna yang mengubah \textit{tab} dari SharIF Judge ke \textit{tab} lain dalam browser.
    \item \verb|pdf_viewer.focusin|: Peristiwa ini akan menangkap pengguna pada saat pengguna \textit{focus} pada PDF Viewer dalam IDE.
    \item \verb|pdf_viewer.focusout|: Peristiwa ini akan menangkap pengguna pada saat pengguna tidak \textit{focus} pada PDF Viewer dalam IDE.
    \item \verb|editor_input.input|: Peristiwa ini akan menangkap perubahan isi editor input dalam IDE. 
    \item \verb|editor_output.change|: Peristiwa ini akan menangkap hasil output saat terjadinya aksi \textit{Save \& Execute} dalam IDE.
    \item \verb|save.click|: Peristiwa ini akan menangkap aksi \textit{Save} yang dilakukan pengguna.
    \item \verb|execute.click|: Peristiwa ini akan menangkap aksi \textit{Save \& Execute} yang dilakukan pengguna.
    \item \verb|submit.click|: Peristiwa ini akan menangkap aksi \textit{Save \& Submit} yang dilakukan pengguna.
\end{itemize}

Seluruh alur sistem perekaman peristiwa dalam SharIF-Judge akan ditambahkan ke dalam file \verb|assets/js/|\verb|shj_submit.js|. Kode perubahan terdapat pada Lampiran ??.
% TODO: Add lampiran perubahan shj_submit

\subsubsection{Perbaikan Implementasi}

Pada saat mengujian fungsi penyimpanan rekaman, dalam tahap alur penyimpanan data rekaman dan juga bagaimana inisialisasi perekaman akan diubah dari perancangan Bab \ref{sub:4:3:merekam}. Hal ini dikarenakan saat pengguna memilih ulang masalah atau memuat ulang halaman Submit, maka semua events yang sudah direkam akan hilang. Maka dari itu berikut merupakan perubahan pada alur fitur sistem perekaman ketikan:

\begin{enumerate}
    \item Inisialisasi Perekam \\
    Alur inisialisasi akan diubah agar dapat memuat events yang sebelumnya sudah disimpan dalam \textit{object javascript} bernama \verb|befRecording|. Oleh karena itu, dibutuhkannya penambahan fungsi pada \textit{Controller} \verb|Submit.php| yaitu fungsi untuk mengambil data rekaman sebelumnya bernama \verb|load_rec| yang mengambil argumen pengenal masalah yang dipilih oleh pengguna. Kode penambahan terdapat di Lampiran ??.
    % TODO: Add lampiran perubahan Submit.php
    \item Penyimpan data rekaman \\
    Pada \textit{object} \verb|recording|, \verb|startValue| dan \verb|startSelection| tidak dibutuhkan karena isi awal dari editor kode dan juga posisi awal cursor dalam editor kode akan selalu berisi dengan nilai kosong yaitu teks kosong dan posisi di baris dan kolom pertama.
\end{enumerate}

\subsection{Menyimpan Rekaman pada Sistem}
\label{sub:5:2:storerekaman}

Fitur penyimpanan rekaman pada sistem bertujuan untuk menyimpan data secara permanen ke dalam database dan \textit{server} agar dapat diputar kembali di lain waktu. Berikut merupakan alur fitur penyimpanan rekaman pada sistem:

\begin{enumerate}
    \item Mengirimkan Data ke \textit{Server} \\
    Dalam halaman Submit, pengguna memiliki 3 aksi penting dalam IDE SharIF-Judge yaitu: \textit{save}, \textit{execute}, dan \textit{submit}. Alur ini akan mengirimkan data rekaman pada saat pengguna melakukan aksi tersebut. Data yang dikirim merupakan \textit{object} \verb|recording| yang sudah jadikan sebagai teks JSON dengan menggunakan fungsi \verb|JSON.stringify|.

    \item Menyimpan Data Dalam File Sistem \\
    File akan disimpan dalam \textit{folder} yang sama dengan penyimpanan kode \textit{submission} yang dijelaskan pada Bab \ref{sub:3:1:penyimpanankode}. File akan diisi secara langsung oleh data rekaman dan tidak diubah oleh \textit{server}. Penamaan file dapat dibagi berdasarkan aksi yang membuat pengguna mengirimkan data rekaman. Untuk aksi \textit{save} dan \textit{execute}, file dengan data rekaman akan disimpan dengan nama \verb|recording|. Untuk aksi \textit{submit}, file akan disimpan dengan nama \verb|recording| dilanjutkan dengan sebuah `-' dan \textit{submit id} yang dibuat. File tersebut akan memiliki tipe data yang sama yaitu JSON dikarenakan itu extensi file yang digunakan adalah \verb|.json|.
 
    \item Menyimpan Data Dalam Database \\
    Saat penyimpanan data ke dalam file sistem berhasil, maka penyimpanan kedalam database juga akan dilakukan. Data yang akan disimpan kedalam database akan digunakan untuk mendaftar rekaman yang ada dalam sistem, maka data yang akan disimpan bukan data rekaman melainkan data statistik. Berikut merupakan data yang akan disimpan ke dalam Database:
    \begin{itemize}
        \item \verb|rec_id|: pengenal rekaman yang sama dengan \textit{submit id}.  
        \item \verb|username|: nama pengguna yang mengirimkan data rekaman.
        \item \verb|problem_id|: pengenal masalah yang pengguna kerjakan.
        \item \verb|assignment_id|: pengenal tugas yang pengguna kerjakan.
        \item \verb|upload_at|: waktu sistem penyimpan data rekaman.
    \end{itemize}
    Untuk membuat databasenya sendiri, dibutuhkan penambahan tabel bernama tabel \verb|recording| yang memiliki lima atribut diatas. Kode \ref{kode:5:2:adddatabasetable} menunjukkan pembuatan tabel baru menggunakan \textit{CodeIgniter} dalam SharIF-Judge.

    \begin{lstlisting}[caption={Kode membuat database pada SharIF-Judge}, label={kode:5:2:adddatabasetable}]
// create table 'recording'
$fields = array(
    'rec_id' 	    => array('type' => 'INT', 'constraint' => 11, 'unsigned' => TRUE),	
    'upload_at'		=> array('type' => $DATETIME),
    'assignment' 	=> array('type' => 'SMALLINT', 'constraint' => 4, 'unsigned' => TRUE),
    'problem'       => array('type' => 'SMALLINT', 'constraint' => 4, 'unsigned' => TRUE),
    'username'      => array('type' => 'VARCHAR', 'constraint' => 20),
);
$this->dbforge->add_field($fields);
if (! $this->dbforge->create_table('recording', TRUE))
    show_error("Error creating database table " . $this->db->dbprefix('recording'));
// ADD Unique constraint
$this->db->query(
    "ALTER TABLE {$this->db->dbprefix('recording')}
        ADD CONSTRAINT {$this->db->dbprefix('sruap_unique')} UNIQUE (rec_id, username, assignment, problem);"
);
    \end{lstlisting}

    Mengikuti arsitektural \textit{CodeIgniter}, untuk menambahkan sebuah data ke dalam database perlu menggunakan sebuah \textit{Model}. Oleh karena itu, dibutuhkannya \textit{model} baru bernama \verb|Recording_model.php| yang ditambahkan fungsi \verb|add_recording()| yang memiliki argumen yaitu seluruh data yang ingin disimpan dalam database.

    Untuk aksi \textit{save} dan \textit{execute} dimana tidak adanya \textit{submit id} maka akan dibuat menjadi angka nol (`0') pada pengenal rekamannya atau \verb|rec_id| jika aksinya merupakan \textit{submit}.
\end{enumerate}

Untuk alur pengiriman data ke \textit{server}, dibutuhkannya penambahan kode ke dalam \verb|assets/js/|\\\verb|shj_submit.js|. Kode pembahan terdapat pada Lampiran ??. Agar dapat menyimpan dibutuhkannya perubahan dalam kode \textit{Controller} \verb|Submit.php| pada fungsi \verb|save($type)| dan fungsi \verb|_submit()|. Kode perubahan \verb|Submit.php| terdapat pada Lampiran ??.
% TODO: ref Lampirkan kode perubahan shj_submit dan Submit.php controller

\subsubsection{Perbaikan Implementasi}

Pada saat pengujian yang sama dengan Bab \ref{sub:5:2:merekam}, dibutuhkannya perubahan pada alur pengirimkan data ke server. Dikarenakan \textit{events} yang sudah di save dapat terhapus karena pengguna memilih ulang masalah dan memuat ulang halaman Submit, yang membuat rekaman inisialisasi dan menghapus rekaman lama. Maka dari itu pada saat mengirimkan data \verb|recording|, akan disertakan juga data \verb|befRecording| yang sudah diambil pada saat inisialisasi. Format pengiriman data juga akan berubah dikarenakan adanya rekaman yang lama menjadi sebuah \textit{key} dan \textit{value} karena hanya dua value yang harus disimpan yaitu \textit{events} sebagai \textit{value} dan \textit{startTime} sebagai \textit{key}. Maka format ini menjadi format keseluruhan events yang terjadi dan dapat disatukan dengan format yang sama menggunakan \textit{spread operator} agar seluruh rekaman lama digabungkan. Kode \ref{kode:5:2:mergedata} merupakan kode untuk mengirimkan teks JSON dengan mengabungkan kedua rekaman menggunakan \textit{spread operator}.

\begin{lstlisting}[caption={\textit{object callback function}}, label={kode:5:2:mergedata}]
JSON.stringify({
    ...befRecording,
    [recording.startTime]: recording.events
}),
\end{lstlisting}

\subsection{Melihat Daftar Rekaman}

Fitur ini digunakan untuk melihat daftar rekaman mahasiswa yang tersimpan dalam sistem, pengguna juga dapat melihat isi rekaman yang terdapat dalam daftar rekaman tersebut. Fitur ini dibutuhkan dua tahap untuk diimplementasikan yaitu implementasi pengambilan data dan implementasi menampilkan data dan antarmuka.

\subsubsection{Pengambilan Data Rekaman}

Fitur pengambilan data rekaman digunakan agar bagian depan SharIF-Judge dapat meminta daftar rekaman yang ada pada bagian belakang SharIF-Judge. Oleh karena ini merupakan sebuah halaman baru dalam SharIF-Judge, maka dibutuhkannya sebuah \textit{Controller} baru bernama \verb|Recording.php| yang menampilkan sebuah halaman baru yang dapat diakses melalui rute \verb|/recording/all/| yang mengunakan fungsi baru dalam \verb|Recording_model.php| yaitu fungsi untuk mendapatkan daftar rekaman dinamakan \verb|all_user_recordings|. Kode pertambahan pada \textit{Model} \verb|Recording.php| berada pada Lampiran ??.
% TODO: Add lampiran kode model recording

Berikut fungsi yang akan diimplementasikan dalam \textit{Controller} \verb|Recording.php|:
\begin{enumerate}
    \item \verb|__construct()| \\
    Fungsi ini akan memuat seluruh kebutuhan \textit{Model} dan \textit{Helper} ke dalam \verb|Recording.php|, fungsi \textit{construct} juga akan membatasi akses oleh pengguna dibawah \textit{instructor}. Fungsi ini juga mendapatkan \textit{params url} yang dikirim oleh pengguna. 

    \item \verb|all()| \\
    Fungsi ini akan mengambil beberapa data yang dibutuhkan oleh antarmuka SharIF-Judge yaitu \textit{assignment} yang dipilih oleh pengguna menggunakan \verb|assignment_model|, \textit{problem} yang ada dalam \textit{assignment} tersebut menggunakan \verb|assignment_model|, dan daftar \textit{recording} yang tersimpan dalam sistem menggunakan \verb|recording_model|. Setelah itu, server akan menempatkan seluruh data tersebut ke dalam \textit{view} baru bernama \verb|recording_list.twig|.
\end{enumerate}

Seluruh alur untuk mengimplementasikan fitur pengambilan data rekaman dalam \textit{Controller} \verb|Recording.php| terdapat dalam Lampiran ??.
% TODO: Add lampiran kode controller recording

\subsubsection{Antarmuka dan Tampilan Data}

Antarmuka yang akan dibuat serupa dengan perancangan pada Bab \ref{sub:4:1:pemutaranulang} yang diimplementasikan ke dalam SharIF-Judge. Data yang dikirim oleh \textit{Controller} \verb|Recording.php| juga dapat ditampilkan menggunakan \textit{Library twig} tanpa menbutuhkan \textit{javascript} maupun \textit{php} dalam antarmukanya. Gambar ?? menunjukkan implementasi antarmuka beserta data yang terdapat dalam sistem. Untuk kode keseluruhan antarmuka menggunakan \textit{Library twig} dapat dilihat pada Lampiran ??.
% TODO: Add Gambar antarmuka halam recording_list.twig
% TODO: Add lampiran kode recording_list.twig

\subsection{Pemutaran Ulang Rekaman}

Fungsi pemutaran ulang rekaman mengunakan data rekaman yang sudah disimpan dalam sistem untuk menvisualisasikan proses penyelesaian masalah pengguna secara kronologis. Fitur pemutaran ulang ini membutuhkan tiga implementasi antarmuka, implementasi memuat data rekaman, dan implementasi menjalankan rekaman. 

\subsubsection{Implementasi Antarmuka}

Untuk menambahkan sebuah halaman baru dalam SharIF-Judge dibutuhkannya juga fungsi baru pada \textit{Controller} \verb|Recording.php|. Fungsi baru akan dinamakan \verb|index| untuk menampilkan sebuah \textit{view} baru bernama \verb|recording.twig|. Fungsi \verb|index| akan menampilkan halaman baru itu melalui rute \verb|/recording/|. Penamaan \textit{index} itu agar rute tidak memerlukan \verb|/index| pada akhir rute karena jika rute \textit{function} (Bab \ref{sub:2:2:codeigniterurls}) akan otomatis mengarah pada fungsi \verb|index| dalam kelas tersebut.

Fungsi \verb|index| akan mengirim data daftar rekaman pengguna lainnya dalam masalah yang dipilih. Data tersebut akan dipakai oleh \verb|recording.twig| untuk menambahkan daftar rekaman pengguna lainnya pada \textit{assignment} dan \textit{problem} yang sama.

Gambar ?? merupakan antarmuka yang diimplementasikan serupa dengan perancangan pada Bab \ref{sub:4:1:pemutaranulang}.

% TODO: Tambah Gambar antarmuka recording.twig

\subsubsection{Implementasi Memuat Data Rekaman}
\label{ssub:5:2:4:memuatdata}

Data rekaman yang akan diambil sudah disimpan (Bab \ref{sub:5:2:storerekaman}) dalam sistem mengunakan \textit{Controller}. Tetapi data rekaman tidak akan dikirim oleh \textit{Controller} pada saat halaman \textit{recording} dimuat, melainkan mengunakan AJAX pada halaman \textit{recording}. Maka dari itu, butuh fungsi baru pada \textit{Controller} \verb|Recording.php| dan sebuah assets \textit{javascript} baru bernama \verb|shj_function.js|.

Fungsi baru dalam \textit{Controller} \verb|Recording.php| akan dinamakan \verb|download_record| yang memiliki argumen \verb|assignment_id|, \verb|problem_id|, dan \verb|rec_id|. Fungsi tersebut akan mengambil file rekaman dalam file sistem dengan mengunakan argumen untuk mendapatkan lokasi dan nama file rekaman (Bab \ref{sec:4:2:storerekaman}) SharIF-Judge dan mengirimkan file tersebut secara langsung. File juga akan dirikim dengan header \verb|Content-Type: application/json| dan \verb|Content-Disposition: attachment; filename=|\\\verb|"rec.json"|.

Fungsi \verb|download_record| akan dipanggil pada saat halaman \textit{recording} dimuat oleh \textit{javascript} \verb|shj_function.js| mengunakan fungsi baru yaitu \verb|getRecording|. Fungsi \verb|getRecording| akan meninisialisasi editor kode dalam antarmuka dan menformat data rekaman agar lebih mudah untuk di putar ulang. Berikut merupakan format data tambahan yang akan diubah oleh fungsi \verb|getRecording|:

\begin{itemize}
    \item \verb|events|: Data rekaman
    \item \verb|eventsIndex|: sebuah map dengan \textit{key} waktu saat sebuah \textit{events} terjadi dan \textit{value index} waktu saat sebuah \textit{events} terjadi.
    \item \verb|indexEvents|: sebuah map dengan \textit{key index} waktu saat sebuah \textit{events} terjadi dan \textit{value} waktu saat sebuah \textit{events} terjadi.
    \item \verb|presumIndexDuration|: menkalkulasikan panjang rekaman sebelum rekaman selesai.
    \item \verb|length|: panjang data rekaman
    \item \verb|duration|: durasi dari seluruh rekaman yang ada pada data rekaman
\end{itemize}

Data tersebut akan disimpan dalam sebuah \textit{object javascript} yang dinamakan \verb|recording| dan akan dipakai pada saat menjalankan rekaman dan untuk menampilkan histogram \textit{events} yang terjadi.

\subsubsection{Implementasi Menjalankan Rekaman}
 
Fitur menjalankan rekaman akan menggunakan data bedasarkan data yang didapatkan oleh AJAX yang dijelaskan pada bagian \ref{ssub:5:2:4:memuatdata}. Fitur menjalankan rekaman akan membutuhkan penambahan kode pada \textit{javascript} \verb|shj_function.js| yang akan menjalankan fungsi \verb|play| atau \verb|stop| untuk menjalankan atau memberhentikan rekaman oleh pengguna. 

Fungsi menjalankan atau mematikan rekaman dibagi menjadi dua yaitu fungsi rekaman dalam IDE dengan data rekaman dinamakan \verb|Recording| dan fungsi timer yang digunakan untuk memberitahu kepada pengguna progress waktu pemutaran rekaman dinamakan \verb|Timer|.
Fungsi \verb|Recording| menggunakan fungsi dalam \textit{Library Ace} dan fungsi \textit{javascript} untuk memperbaharui IDE antarmuka berdasarkan \textit{event} yang dipanggil, fungsi \verb|setTimeout| dalam \textit{javascript} akan digunakan untuk menjalankan \textit{event} selanjutnya bedasarkan perbedaan waktu antara event sekarang dan event selanjutnya dengan memanggil fungsi \verb|playRecording| dengan \textit{event} selanjutnya. Sedangkan fungsi \verb|Timer| menggunakan fungsi \verb|setInterval| yang akan dijalankan berulang untuk setiap detiknya dan memperbaharui progress waktu dalam antarmuka berdasarkan waktu yang sudah lewat.

\subsubsection{Menampilkan Histogram Events yang Terjadi}

Pada fungsi \verb|getRecording| setelah memuat data rekaman dan menformat data rekaman tersebut, fungsi \verb|setUpChart| akan dipanggil dan membuat data grafik histogram. Data histogram akan dimuat menggunakan \textit{Library} \verb|Chart.js|.


\section{Pengujian Fungsional}
\label{sec:5:fungsional}

\section{Pengujian Eksperimental}
\label{sec:5:eksperimental}

Pada Bagian ini dilakukannya pengujian terhadap sistem Perekaman Ulang pada SharIF Judge.

\subsection{Lingkungan pengujian}
Pengujian sistem perekaman ulang akan dilakukan pada sebuah server VPS atau \textit{Virtual Private Server}. Pada server akan dijalankannya \textit{docker} agar sistem aplikasi akan identik dengan pada saat sistem sedang diimplementasikan. Kode \ref{kode:5:3:1:dockercompose} dan Kode \ref{kode:5:3:1:dockerfile} merupakan file \textit{docker-compose} dan \textit{Dockerfile} yang digunakan membangun sistem SharIF Judge dalam server VPS.

\begin{lstlisting}[caption=File \textit{docker-compose} yang digunakan, label=kode:5:3:1:dockercompose]
version: "3"
services:
  codeigniter-3:
    build: .
    ports:
      - "81:80"
    volumes:
      - .:/var/www/html
    depends_on:
      - db

  db:
    image: mysql:5.7
    ports:
      - "3306:3306"
    environment:
      MYSQL_TABLE: judge
      MYSQL_USER: sharif
      MYSQL_PASSWORD: judge
      MYSQL_ROOT_PASSWORD: root
    command: --sql_mode=STRICT_TRANS_TABLES,NO_ZERO_IN_DATE,NO_ZERO_DATE,ERROR_FOR_DIVISION_BY_ZERO,NO_ENGINE_SUBSTITUTION
    volumes:
      - ./mysql:/var/lib/mysql

  phpmyadmin:
    image: phpmyadmin/phpmyadmin
    ports:
      - "8080:80"
    environment:
      PMA_HOST: db
      MYSQL_ROOT_PASSWORD: freehost
    depends_on:
      - db
\end{lstlisting}

\begin{lstlisting}[caption=File \textit{Dockerfile} yang digunakan, label=kode:5:3:1:dockerfile]
# Menggunakan image PHP 7.3 sebagai base image
FROM php:7.3-apache

# Install dependensi dan ekstensi PHP yang dibutuhkan untuk CodeIgniter
RUN apt-get update && apt-get install -y \
    libpng-dev \
    libjpeg-dev \
	libldap2-dev \
	libcurl4 \
    libcurl4-openssl-dev \
	libzip-dev \
    libfreetype6-dev \
    zip \
    unzip \
	default-jdk \
	g++ \
	python2 \
	python3

# Install ekstensi GD dan mysqli
RUN docker-php-ext-configure gd --with-freetype-dir=/usr/include/ --with-jpeg-dir=/usr/include/ \
    && docker-php-ext-install gd mysqli
	
RUN docker-php-ext-install curl

RUN  docker-php-ext-configure ldap --with-libdir=lib/x86_64-linux-gnu/ \
	&& docker-php-ext-install ldap

RUN docker-php-ext-install fileinfo
RUN docker-php-ext-install mbstring
RUN docker-php-ext-install zip

RUN cp /usr/local/etc/php/php.ini-production /usr/local/etc/php/php.ini && \
        sed -i -e "s/^ *memory_limit.*/memory_limit = 4G/g" /usr/local/etc/php/php.ini && \
        sed -i -e "s/^ *max_input_vars.*/max_input_vars = 3000000/g" /usr/local/etc/php/php.ini && \
        sed -i -e "s/^ *post_max_size.*/post_max_size = 50M/g" /usr/local/etc/php/php.ini && \
        sed -i -e "s/^ *upload_max_filesize.*/upload_max_filesize = 50M/g" /usr/local/etc/php/php.ini

# Aktifkan mod_rewrite untuk Apache
RUN a2enmod rewrite

# Copy kode CodeIgniter ke dalam container
COPY . /var/www/html/

# Set direktori kerja
WORKDIR /var/www/html/

# Make Folder tester writeable by PHP
RUN chmod 777 /var/www/html/restricted/tester
RUN chmod 777 /var/www/html/application/cache/Twig

# Expose port 80
EXPOSE 80

# Jalankan Apache server
CMD ["apache2-foreground"]
\end{lstlisting}

Tabel \ref{tab:5:1:keraspembangunan} menunjukkan spesifikasi perangkat keras yang digunakan saat pengujian.
\begin{table}[H]
    \caption{Perangkat Keras Lingkungan Pembangunan}
    \label{tab:5:1:keraspembangunan}
    \centering
    \begin{tabular}{|l|l|}
        \hline
        \textbf{Parameter}                  & \textbf{Nilai}              \\ \hline
        \textit{Processor}                  & \textit{AMD EPYC 9354P 2 vCPU} \\ \hline
        \textit{Random Access Memory (RAM)} & 8 GB                       \\ \hline
        \textit{Storage}                    & 100 GB \textit{SSD}           \\ \hline
    \end{tabular}
\end{table}

Tabel \ref{tab:5:2:1:lunakpembangunan} menunjukan spesifikasi perangkat lunak yang digunakan saat pengujian.

\begin{table}[H]
    \caption{Perangkat Lunak Lingkungan Pembangunan}
    \label{tab:5:2:1:lunakpembangunan}
    \centering
    \begin{tabular}{|l|l|}
        \hline
        \textbf{Parameter}        & \textbf{Nilai}                                            \\ \hline
        Sistem Operasi            & \textit{Debian Version} 12     \\ \hline
        Bahasa Pemrograman        & PHP, \textit{JavaScript}, \textit{CSS}, dan \textit{HTML} \\ \hline
        \textit{Framework}        & \textit{CodeIgniter} 3.1.13                               \\ \hline
        \textit{Code Editor}      & \textit{Visual Studio Code} 1.99.3                        \\ \hline
        Perangkat Lunak Pendukung & \textit{Docker Version} 20.10.24+dfsg1                            \\ & \textit{Debian} 11-slim \\ & \textit{MySQL} 5.7 \\ & \textit{phpMyAdmin} 5.2.1 \\ & PHP 7.3.33\\ \hline
    \end{tabular}
\end{table}

\subsection{Pengujian}

Pengujian dilakukan untuk mengetahui permasalah yang terjadi jika IDE dalam SharIF Judge dimasukkan sistem perekaman. Hasil Pengujian juga akan dianalisis secara sederhana agar dapat dibuatnya sebuah sistem untuk mendeteksi tindakan kecurangan.

Pengujian pada sistem pemutaran ulang dilakukan dengan cara mengajak beberapa peserta yang masih menempuh kuliah maupun yang sudah lulus dengan mengerjakan tiga permasalahan dengan tingkat kesusahan mudah, sedang, dan sulit dalam waktu tempuh satu hari. Para peserta dianjurkan mengerjakan 2 soal yang sudah disediakan dan dapat melihat \textit{syntax} bahasa pemrograman. Pada pengujian peserta juga dapat melakukan kecurangan pada satu buah nomor dengan cara apapun. Kecurangan akan dianggap terjadi pada saat peserta melihat cara pengerjaan permasalah dengan cara apapun.

Berikut merupakan masalah yang ditemukan pada saat pengujian sistem pemutaran ulang:

\begin{itemize}
    \item Fitur menampilkan soal pada IDE \\
    Fitur ini tidak dapat dilakukan karena permasalahan dengan \textit{networking} dalam server VPS yang menjadikan tidak bisanya mengakses PDF permasalah dalam SharIF Judge dan menghasilkan \textit{status} 404. Fitur ini diabaikan karena fitur ini tidak mengganggu pengujian yang sedang berjalan, tetapi untuk peserta melihat soal dengan cara mendownload soal dan melihatnya pada aplikasi lain. 
    \item Fitur Menyimpan Rekaman pada Sistem \\
    Fitur ini tidak mengubah database pada saat pengguna melakukan aksi \textit{submit}. Pada saat aksi \textit{submit} dilakukan file rekaman dibuah menjadi file rekaman yang sudah di-\textit{submit} tetapi pada daftar rekaman dalam database itu sendiri belum dihapus dan pada saat rekaman di ambil, tidak ditemukannya rekaman tersebut dan mengembalikan error pada pengguna. Untuk menyelesaikan masalah ini, diperlukannya sebuah fungsi baru dalam \textit{Model} \verb|Recording_model|. Kode \ref{kode:5:2:2:recording_model} merupakan fungsi yang ditambahkan pada \textit{Model} \verb|Recording_model|. 

    \begin{lstlisting}[language=php, caption={Fungsi tambahan pada \textit{Recording model}}, label=kode:5:2:2:recording_model]
public function remove_saveonly_recording($assignment_id, $problem_id, $username) {
    $this->db->delete('recording', array(
        'assignment' => $assignment_id, 
        'problem'=>$problem_id, 
        'username'=>$username, 
        'rec_id'=>0)
    );
}
    \end{lstlisting}

    Penambahan fungsi ini akan digunakan pada \textit{Controller} \verb|Recording| pada saat fungsi \verb|_submit()| dipanggil.

\end{itemize}

Setelah pengujian berakhir, peserta diminta untuk mengisi beberapa pertanyaan mengenai permasalahan yang dikerjakan dan juga beberapa pertanyaan mengenai pengalaman mengerjakan permasalah pemrograman. 

\subsubsection{Analisa Hasil Pengujian}
Analisa ini dilakukan dengan melihat data yang rekaman yang sudah di proses dan menjadi bagan histogram dalam sistem dan bagan lainnya yang dibutuhkan pada proses analisa. Ada beberapa pola yang dapat dilihat pada saat melihat histogram hasil rekaman peserta pengujian. Berikut merupakan pola yang dapat dilihat pada:

\begin{itemize}
    \item Pola Pembuatan Kode
    \item Pola \textit{Debugging} \\
    Pola ini dapat dilihat pada \textit{events} \verb|editor.change|, \verb|editor.changeCursor|, \verb|editor.change|\\\verb|Selection|, \verb|editor_input.input|, dan \verb|editor_output.change|. Pengguna yang \textit{debugging} akan mengubah kode pada lokasi yang sama. Gambar \ref{fig:5:2:3:heatmapchange} menunjukkan bahwa frekuensi lebih tinggi akan berwarna lebih merah dan menandakan pola debugging yang dapat dilihat karena terjadinya banyak perubahan pada posisi yang sama dibandingkan dengan Gambar \ref{fig:5:2:3:heatmapchangenodebug} yang memiliki frekuensi maksimum 12 pada warna termerahnya.

    \begin{figure}[H]
        \centering
        \includegraphics[width=0.75\textwidth]{experiment/change-heatmap.png}
        \caption{Bagan Heatmap Perubahan Saat \textit{Debugging}}
        \label{fig:5:2:3:heatmapchange}
    \end{figure}

    \begin{figure}[H]
        \centering
        \includegraphics[width=0.75\textwidth]{experiment/change-heatmap-nodebug.png}
        \caption{Bagan Heatmap Perubahan Tanpa \textit{Debugging}}
        \label{fig:5:2:3:heatmapchangenodebug}
    \end{figure}
    
    Peserta yang \textit{debugging} juga akan mencoba untuk mengubah input dan melakukan aksi \textit{Execute} lebih banyak dibandingkan dengan peserta yang tidak melakukan \textit{debugging}. Gambar \ref{fig:5:2:3:debugmany} menunjukkan warna kuning sebagai perubahan input dan warna biru sebagai aksi \textit{Execute} yang dilakukan sedangkan Gambar \ref{fig:5:2:3:debugless} menunjukkan warna biru sebagai perubahan input dan warna merah sebagai aksi \textit{Execute} yang dilakukan. Perbedaan kedua bagan adalah frekuensi dimana aksi \textit{Execute} dilakukan dan juga perubahan input. Maka pola \textit{debugging} juga dapat dilihat melalui frekuensi aksi \textit{Execute} dan juga frekuensi perubahan input.

    \begin{figure}[H]
        \centering
        \includegraphics[width=0.75\textwidth]{experiment/debugging-many-io.png}
        \caption{Bagan Histogram Perubahan Saat \textit{Debugging}}
        \label{fig:5:2:3:debugmany}
    \end{figure}

    \begin{figure}[H]
        \centering
        \includegraphics[width=0.75\textwidth]{experiment/debugging-less-io.png}
        \caption{Bagan Histogram Perubahan Tanpa \textit{Debugging}}
        \label{fig:5:2:3:debugless}
    \end{figure}

    \item Pola Distraksi
    \item Pola Perubahan Navigasi
    \item Pola Berpikir
    \item Pola \textit{Copy-Paste}
    \item Pola Perubahan Cursor dalam Editor Kode
\end{itemize}