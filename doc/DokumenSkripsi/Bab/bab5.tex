\chapter{Implementasi dan Pengujian}
\label{chap:implementasidanpengujian}

Bab ini membahas mengenai implementasi dan pengujian sistem perekaman ulang dalam SharIF-Judge.

\section{Lingkungan Implementasi dan Pengujian}
\label{sec:5:lingkungan}

Implementasi perangkat lunak sistem perekaman ulang dilakuka npada dua buah lingkungan. Lingkungan pertama digunakan untuk membangun perangkat lunak sedangkan lingkungan kedua digunakan untuk melakukan pengujian. Berikut merupakan spesifikasi lingkungan implementasi dan pengujian yang digunakan:

\begin{itemize}
    \item Lingkungan Pembangunan \\
          Tabel \ref{tab:5:1:keraspembangunan} menunjukkan spesifikasi perangkat keras yang digunakan saat pembangunan.
          \begin{table}[H]
              \caption{Perangkat Keras Lingkungan Pembangunan}
              \label{tab:5:1:keraspembangunan}
              \centering
              \begin{tabular}{|l|l|}
                  \hline
                  \textbf{Parameter}                  & \textbf{Nilai}        \\ \hline
                  Perangkat Keras                     & Asus ROG Zephyrus M16 \\ \hline
                  \textit{Processor}                  & \textit{i9-6950H}     \\ \hline
                  \textit{Random Access Memory (RAM)} & 16 GB                 \\ \hline
                  \textit{Storage}                    & 1 TB \textit{SSD}     \\ \hline
              \end{tabular}
          \end{table}

          Tabel \ref{tab:5:1:lunakpembangunan} menunjukan spesifikasi perangkat lunak yang digunakan saat pembangunan.

          \begin{table}[H]
              \caption{Perangkat Lunak Lingkungan Pembangunan}
              \label{tab:5:1:lunakpembangunan}
              \centering
              \begin{tabular}{|l|l|}
                  \hline
                  \textbf{Parameter}        & \textbf{Nilai}                                            \\ \hline
                  Sistem Operasi            & \textit{Windows 11 Version} ???                           \\ \hline
                  Bahasa Pemrograman        & PHP, \textit{JavaScript}, \textit{CSS}, dan \textit{HTML} \\ \hline
                  \textit{Framework}        & \textit{CodeIgniter} 3.1.13                               \\ \hline
                  \textit{Code Editor}      & \textit{Visual Studio Code} 1.99.3 (Universal)            \\ \hline
                  Perangkat Lunak Pendukung & \textit{Docker Version} ???                               \\ & \textit{Debian} 11-slim \\ & \textit{Google Chrome Version} ??? (Official Build) (arm64)\\ & \textit{MariaDB} ??? \\ & \textit{phpMyAdmin} ??? \\ & PHP 7.?\\ \hline
              \end{tabular}
          \end{table}

    \item Lingkungan Eksperimental
          %   TODO: Catat lingkungan yang dipakai saat eksperimen
\end{itemize}

\section{Implementasi}
\label{sec:5:implementasi}

\section{Pengujian Fungsional}
\label{sec:5:fungsional}

\section{Pengujian Eksperimental}
\label{sec:5:eksperimental}