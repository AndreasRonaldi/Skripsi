\chapter{Kesimpulan dan Saran}
\label{chap:kesimpulandansaran}

Bab ini akan membahas kesimpulan dari hasil analisis, perancangan, implementasi, pengujian, dan saran-saran untuk pengembangan yang dapat dipertimbangkan.

\section{Kesimpulan}
Berdasarkan proses dan hasil analisis, perancangan, implementasi, dan pengujian perangkat lunak yang telah dibuat, maka diperoleh kesimpulan sebagai berikut:

\begin{enumerate}
	\item Implementasi fitur merekam dan memutar ulang ketikan pada SharIF judge berhasil dikembangkan dengan memanfaatkan fitur dalam \textit{CodeIgniter}, \textit{library twig} sebagai tampilan, \textit{library ace} sebagai editor kode, \textit{library Chart.js} sebagai pembuat bagan dan fitur pada \textit{JavaScript}.

	\item Pemutaran ulang rekaman dapat memberikan gambaran yang jelas mengenai proses berpikir mahasiswa, termasuk aktivitas seperti pengetikan kode, eksekusi program, perubahan tab, hingga pola interaksi lainnya dalam menyelesaikan soal. Dengan adanya rekaman ini, dosen dapat menilai apakah mahasiswa menyelesaikan tugasnya secara mandiri atau tidak.
	
	\item Eksperimen membuktikan bahwa pola pengerjaan mahasiswa yang normal dan yang terindikasi melakukan kecurangan dapat dibedakan melalui histogram interaksi dan analisis terhadap perilaku seperti copy-paste kode, kecepatan ketikan, frekuensi berpikir (pause sebelum mengetik ulang), dan intensitas debugging.

	\item Analisis data rekaman menunjukkan potensi besar dalam mendeteksi kecurangan secara manual, walaupun sistem belum secara otomatis memberikan penilaian atau keputusan. Hal ini membuka peluang untuk pengembangan sistem pendeteksian otomatis berbasis pola.
	
	\item Sistem ini dibatasi pada aktivitas yang terekam dalam halaman IDE, sehingga kecurangan di luar sistem seperti menggunakan perangkat lain, tidak dapat dideteksi. Namun demikian, sistem ini sudah menyediakan data yang cukup kaya untuk dilakukan analisis awal terhadap dugaan kecurangan.
\end{enumerate}

\section{Saran}
Berdasarkan proses dan hasil analisis, perancangan, implementasi, dan pengujian perangkat lunak yang telah dibuat, maka diperoleh saran-saran sebagai berikut:

\begin{enumerate}
	\item Pengembangan fitur analisis otomatis terhadap pola interaksi, yang dapat memberikan notifikasi kepada dosen apabila ditemukan indikasi kuat terhadap kecurangan, misalnya pola copy-paste mendadak.
	\item Optimalisasi penyimpanan rekaman, mengingat setiap aktivitas mahasiswa disimpan dalam bentuk JSON yang dapat menjadi besar. Kompresi data atau sistem penyimpanan dapat dipertimbangkan agar lebih efisien.
	\item Pada saat pengujian, banyak peserta yang menyampaikan keluhan mengenai perilaku IDE terutama untuk mengetahui kesalahan dalam kode program berbahasa \textit{Python}. Maka IDE dapat dioptimasi agar kesalahan dalam kode program dapat dimunculkan.
	\item Menambahkan sebuah pilihan pada saat pembuatan tugas atau \textit{assignment} untuk menyalakan \textit{auto-complete} kode dalam SharIF Judge.
	\item Perluasan cakupan rekaman aktivitas, misalnya dengan merekam tangkapan layar atau memperluas deteksi tab yang dibuka pengguna, selama masih dalam ranah yang etis dan tidak melanggar privasi mahasiswa.
	\item Melanjutkan eksperimen dengan skala yang lebih besar, agar dapat ditentukan batas-batas kuantitatif yang lebih akurat dalam mengidentifikasi pola-pola interaksi mahasiswa yang mencurigakan.
\end{enumerate}