\chapter{Kesimpulan dan Saran}
\label{chap:kesimpulandansaran}

Bab ini akan membahas kesimpulan dari hasil analisis, perancangan, implementasi, pengujian, dan saran-saran untuk pengembangan yang dapat dipertimbangkan.

\section{Kesimpulan}
Berdasarkan proses dan hasil analisis, perancagan, implementasi, dan pengujian perangkat lunak yang telah dibuat, maka diperoleh kesimpulan sebagai berikut:

\begin{enumerate}
	\item Implementasi fitur merekam dan memutar ulang ketikan pada SharIF judge berhasil dikembangkan dengan memanfaatkan fitur \textit{Library ace}, dan fitur pada javascript.
	
	\item Fitur pemutaran ulang ini dapat membantu dosen dalam menganalisa proses berpikir mahasiswa saat menyelesaikan sebuah masalah pemrograman dan menjadi bukti indikasi terjadinya kecurangan.
\end{enumerate}

\section{Saran}
Berdasarkan proses dan hasil analisis, perancagan, implementasi, dan pengujian perangkat lunak yang telah dibuat, maka diperoleh saran-saran sebagai berikut:

\begin{enumerate}
	\item Menambahkan analisis otomatis untuk mendeteksi pola kecuranga dari data ketikan.
	\item Optimasi penyimpanan data rekaman dengan kompresi data.
	\item Pada saat pengujian, banyak peserta yang menyampaikan keluhan mengenai perilaku IDE terutama untuk mengetahui kesalahan dalam kode program berbahasa \textit{python}. Maka IDE dapat dioptimasi agar kesalahan dalam kode program dapat dimunculkan.
	\item Menambahkan sebuah pilihan pada saat pembuatan tugas atau \textit{assignment} untuk menyalakan \textit{auto-complete} kode dalam SharIF Judge.
\end{enumerate}