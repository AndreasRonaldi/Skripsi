\documentclass[a4paper,twoside]{article}
\usepackage[T1]{fontenc}
\usepackage[bahasa]{babel}
\usepackage{graphicx}
\usepackage{graphics}
\usepackage{float}
\usepackage[cm]{fullpage}
\pagestyle{myheadings}
\usepackage{etoolbox}
\usepackage{setspace} 
\usepackage{lipsum} 
\setlength{\headsep}{30pt}
\usepackage[inner=2cm,outer=2.5cm,top=2.5cm,bottom=2cm]{geometry} %margin
% \pagestyle{empty}

\makeatletter
\renewcommand{\@maketitle} {\begin{center} {\LARGE \textbf{ \textsc{\@title}} \par} \bigskip {\large \textbf{\textsc{\@author}} }\end{center} }
\renewcommand{\thispagestyle}[1]{}
\markright{\textbf{\textsc{AIF184001/AIF184002 \textemdash Rencana Kerja Tugas Akhir \textemdash Sem. Ganjil 2023/2024}}}

\newcommand{\HRule}{\rule{\linewidth}{0.4mm}}
\renewcommand{\baselinestretch}{1}
\setlength{\parindent}{0 pt}
\setlength{\parskip}{6 pt}

\onehalfspacing
 
\begin{document}

\title{\@judultopik}
\author{\nama \textendash \@npm} 

%tulis nama dan NPM anda di sini:
\newcommand{\nama}{Andreas Ronaldi}
\newcommand{\@npm}{6182101026}
\newcommand{\@judultopik}{Pemutar Ulang Ketikan Mahasiswa pada SharIF Judge } % Judul/topik anda
\newcommand{\jumpemb}{1} % Jumlah pembimbing, 1 atau 2
\newcommand{\tanggal}{01/01/1900}

% Dokumen hasil template ini harus dicetak bolak-balik !!!!

\maketitle

\pagenumbering{arabic}

\section{Deskripsi}
Salah satu cara untuk penilaian kode yang ditulis oleh mahasiswa adalah dengan menilainya pada sebuah website judge. Judge ini dapat menilai hasil keluaran sebuah kode dari berbagai input dan output yang disesuaikan oleh dosen/pengajar. Cara penilaian ini digunakan oleh Teknik Informatika UNPAR untuk menilai hasil kode dari para mahasiswanya. Judge yang digunakan adalah SharIF-Judge dimodifikasi oleh Stillmen Vallian terhadap Sharif-Judge buatan Mohammad Javad Naderi dengan \textit{framework} CodeIgniter dan Bash.

Skripsi PAN5001 menceritakan bahwa judge ini tidak memiliki kemampuan untuk mengawasi proses pembuatan kode program karena para mahasiswa menggunakan aplikasi eksternal untuk pembuatan kode program tersebut. Sehingga dibuatnya modifikasi terhadap SharIF-Judge untuk menambahkan \textit{Intergrated Development Enviroment} (IDE), sebuah aplikasi untuk mengedit, mengompilasi, dan menjalankan kode program pada SharIF-Judge dengan \textit{framework} Ace.

Pada tugas akhir ini, akan dibuat kode rekaman di IDE yang tersedia pada SharIF-Judge untuk membantu pengawasan dengan merekan dan memutar ulang ketikan di IDE. Tugas ini akan membuat pengawasan terhadap kegiatan kuliah lebih mudah untuk pengawas dan dapat menjadi bukti kecurangan jika dibutuhkan.

\section{Rumusan Masalah}
Rumusan Masalah yang akan di bahas pada Skripsi ini adalah:
 \begin{enumerate}
     \item Bagaimana agar kode editor lebih mudah untuk dipakai oleh mahasiswa.
     \item Bagaimana menginplementasikan pemutaran ulang ketikan mahasiswa pada IDE SharIF-Judge.
     \item Bagaimana cara menyimpan data pemutaran ulang mahasiswa secara rutin dengan otomatis dan tidak mengambil storage database sangat besar.
 \end{enumerate}

\section{Tujuan}
Tujuan yang ingin dicapai skripsi ini adalah sebagai berikut:
    \begin{enumerate}
        \item Mengimplementasikan kode editor mirip dengan aplikasi eksternal yang dipakai mahasiswa.
        \item Mengimplementasikan perekaman dan pemutaran ulang ketikan mahasiswa pada IDE SharIF-Judge.
        \item Mencari cara peminmpanan data efektif dan mengimplementasikannya pada perekaman dan pemutaran ulang ketikan. 
    \end{enumerate}
\section{Deskripsi Perangkat Lunak}

Perangkat lunak akhir yang akan dibuat memiliki fitur minimal sebagai berikut:
\begin{itemize}
	\item SharIF-Judge dapat merekan semua event-event (ketikan, save, load, test) yang terjadi pada IDE.
    \item SharIF-Judge dapat menyimpan semua rekaman mahasiswa.
    \item Dosen dapat menjalankan ulang rekaman ketikan mahasiswa yang terjadi pada IDE SharIF-Judge.

    % ide dari saya yang mungkin bisa membantu user baru lebih tertarik pakai IDE-nya
    \item Pengguna dapat membuat kode editor di halaman baru
    \item Pengguna dapat memilih problem langsung di kode editor-nya
    \item Pengguna dapat merun beberapa input secara bersamaan
		
\end{itemize}

\section{Detail Pengerjaan Tugas Akhir}

Bagian-bagian pekerjaan skripsi ini adalah sebagai berikut :
	\begin{enumerate}

        \item Melakukan studi tentang framework PHP, CodeIgniter dan Ace.
        \item Melakukan studi tentang cara penyimpanan rekaman ketikan
        \item Memodelkan dan merencanakan perubahan pada structur website dan database pada SharIF-Judge Unpar
        \item Mengimplementasikan rekaman ketikan pada SharIF-Judge
        \item Melakukan Pengujian dan eksperimen
        \item Menulis dokumen skripsi tugas akhir
 
	\end{enumerate}

\section{Rencana Kerja}
Rincian capaian yang direncanakan di Tugas Akhir 1 adalah sebagai berikut:
\begin{enumerate}
\item Mempelajari tentang PHP, CodeIgniter 3, Ace
\item Melakukan studi tentang cara penyimpanan rekaman ketikan
\item Memodelkan dan merencanakan perubahan pada structur website dan database pada SharIF-Judge Unpar
\item Menulis sebagian dokumen tugas akhir yaitu bab 1, 2, 3
\end{enumerate}

Sedangkan yang akan diselesaikan di Tugas Akhir 2 adalah sebagai berikut:
\begin{enumerate}
\item Mengimplementasikan rekaman ketikan pada SharIF-Judge
\item Melakukan Pengujian dan eksperimen
\item Melanjutkan penulis dokumen tugas akhir yaitu bab 4, 5, 6 
\end{enumerate}

\vspace{1cm}
\centering Bandung, \tanggal\\
\vspace{2cm} \nama \\ 
\vspace{1cm}

Menyetujui, \\
\ifdefstring{\jumpemb}{2}{
\vspace{1.5cm}
\begin{centering} Menyetujui,\\ \end{centering} \vspace{0.75cm}
\begin{minipage}[b]{0.45\linewidth}
% \centering Bandung, \makebox[0.5cm]{\hrulefill}/\makebox[0.5cm]{\hrulefill}/2013 \\
\vspace{2cm} Nama: \makebox[3cm]{\hrulefill}\\ Pembimbing Utama
\end{minipage} \hspace{0.5cm}
\begin{minipage}[b]{0.45\linewidth}
% \centering Bandung, \makebox[0.5cm]{\hrulefill}/\makebox[0.5cm]{\hrulefill}/2013\\
\vspace{2cm} Nama: \makebox[3cm]{\hrulefill}\\ Pembimbing Pendamping
\end{minipage}
\vspace{0.5cm}
}{
% \centering Bandung, \makebox[0.5cm]{\hrulefill}/\makebox[0.5cm]{\hrulefill}/2013\\
\vspace{2cm} Nama: Pascal Alfadian Nugroho, S.Kom., M.Comp.\\ Pembimbing Tunggal
}
\end{document}
